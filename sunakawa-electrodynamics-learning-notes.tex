\documentclass[a4j,12pt,uplatex,dvipdfmx]{jsreport}
\usepackage{physics2}
\usepackage{amsmath}
\usepackage{bm}

% セクション番号の表記を砂川理論電磁気に揃える
\renewcommand{\thesection}{\S \arabic{section}}

\title{砂川理論電磁気学 学習ノート}
\author{orange-kyoto}
\date{\today}
\begin{document}
\maketitle

\tableofcontents
\clearpage


\chapter{真空電磁場の基本法則}

\section{場の概念}

\section{電場と磁場の定義}

\section{Coulomb の法則}

閉曲面$S$の外部に点電荷が存在するときは、以下が"容易に"示せるとのこと:

\begin{eqnarray*}
    \oint_{S} \bm{E}(\bm{x}) \cdot \bm{n}(\bm{x}) dS = 0
\end{eqnarray*}

ガウスの定理より、

\begin{eqnarray*}
    \oint_{S} \bm{E}(\bm{x}) \cdot \bm{n}(\bm{x}) dS = \int_{V} \nabla \cdot \bm{E}(\bm{x}) d^3 x
\end{eqnarray*}

閉曲面内に点電荷がないので、電場が無限大になるような特異点が無い。よって、電場のdivを普通に計算することができる。

\begin{eqnarray*}
    \frac{\partial}{\partial x} E_{x} 
    &=& \frac{\partial}{\partial x} \frac{e}{4 \pi \epsilon_0} \frac{x - x_{Q}}{R^3} \\
    &=& \frac{e}{4 \pi \epsilon_0} \left\{ \frac{1}{R^3} - \frac{3 ( x - x_{Q} )^2}{R^5} \right\} 
\end{eqnarray*}

同じように y, z の偏微分も計算できるので

\begin{eqnarray*}
    \nabla \cdot \bm{E}(\bm{x}) &=& \frac{e}{4 \pi \epsilon_0} \left\{ \frac{3}{R^3} - \frac{3 ( x - x_{Q} )^2 + 3 ( y - y_{Q} )^2 + 3 ( z - z_{Q} )^2}{R^5} \right\} \\
    &=& \frac{e}{4 \pi \epsilon_0} \left\{ \frac{3}{R^3} - \frac{3 R^2}{R^5} \right\} \\
    &=& \frac{e}{4 \pi \epsilon_0} \left\{ \frac{3}{R^3} - \frac{3}{R^3} \right\} \\
    &=& 0
\end{eqnarray*}

になり、積分しても$0$になる。電場の湧き出しが無いので、閉曲面を出入りする電場が差し引きでゼロになるイメージ。
\section{Faraday の電磁誘導の法則}

\section{Ampere の法則}

\section{電荷保存則と変位電流}

\section{Maxwell の方程式}

帯電体 $\rho_0$ 自身が作る静電場から受ける力を自己力と呼ぶらしい。次の形でかける\footnote{そういえばこの形だと一見、$\bm{x}$と$\bm{x'}$が近いような領域で分母が発散しそうに思う。そう思った。けど、電荷分布が有限である限りは発散しない。点電荷だと電荷分布が$\rho (x) = q \delta(\bm{x} - \bm{x}_0)$などと書けるので積分が発散してしまうけどね。これはまた別で勉強してまとめてみたい。}:

\begin{eqnarray*}
    \bm{F}^{(e)}_{s} &=& \int_{V} \rho_0(\bm{x}) \bm{E}_0(\bm{x}) d^3 x \\
    &=& \frac{1}{4 \pi \epsilon_0} \int_V d^3 x \int_V d^3 x' \frac{\rho_0(\bm{x}) (\bm{x} - \bm{x'}) \rho_0(\bm{x'})}{| \bm{x} - \bm{x'} |^3}
\end{eqnarray*}

2行目への変形では、帯電体の位置$\bm{x'}$の微小領域の電荷$\rho_0(\bm{x'}) d^3 x'$が位置$\bm{x}$に作る静電場がクーロンの法則からわかっているので、それを重ね合わせた電場の式を使っている。
これが$0$になるとのことだが、以下のように考えると当然のようにも思えてくる。

領域$V$内の2点$\bm{x}$,$\bm{x'}$の微小領域を考え、それぞれに位置する電荷$\rho_0(\bm{x})d^3x$と$\rho_0(\bm{x'})d^3x'$が及ぼし合う力を考えると、当然ながらクーロンの法則より、互いに同じ大きさで逆向きの力を及ぼし合うことになる。

位置$\bm{x}$の電荷が受ける力は、

\begin{eqnarray*}
    \bm{F}_1 = \frac{1}{4 \pi \epsilon_0}  \frac{\rho_0(\bm{x})d^3 x (\bm{x} - \bm{x'}) \rho_0(\bm{x'}) d^3 x'}{| \bm{x} - \bm{x'} |^3}
\end{eqnarray*}

と書けるし、逆に位置$\bm{x'}$の電荷が受ける力は、

\begin{eqnarray*}
    \bm{F}_2 = \frac{1}{4 \pi \epsilon_0}  \frac{\rho_0(\bm{x'}) d^3 x' (\bm{x'} - \bm{x}) \rho_0(\bm{x}) d^3 x}{| \bm{x'} - \bm{x} |^3} = - \bm{F}_1
\end{eqnarray*}

と書ける。よって2点の合力は$\bm{F} = \bm{F}_1 + \bm{F}_2 = 0$になる。
最初の自己力の式は領域$V$のすべての点のペア\footnote{$\bm{x}$,$\bm{x'}$どちらも$V$全体で足し合わせているので、厳密には1つのペアについて2回足し算しているとは思う。が、結局$0$になるのでここでは問題にはならない。自己力をきちんと考慮する必要がある場面ではどう扱うのだろう?それともクーロン力で考えること自体が誤っているのだろうか。}について足し合わせているだけと考えると、合力$0$をいくら足し合わせても全体が$0$なので、自己力が$0$になるのは当然のように思われる。














\end{document}
