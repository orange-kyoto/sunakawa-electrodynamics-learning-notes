\documentclass[a4j,12pt,uplatex,dvipdfmx]{jsreport}
\usepackage{physics2}
\usepackage{amsmath}
\usepackage{bm}

% セクション番号の表記を砂川理論電磁気に揃える
\renewcommand{\thesection}{\S \arabic{section}}

\title{砂川理論電磁気学 学習ノート}
\author{orange-kyoto}
\date{\today}
\begin{document}
\maketitle

\tableofcontents
\clearpage


\chapter{真空電磁場の基本法則}

\section{場の概念}

\section{電場と磁場の定義}

\section{Coulomb の法則}

閉曲面$S$の外部に点電荷が存在するときは、以下が"容易に"示せるとのこと:

\begin{eqnarray*}
    \oint_{S} \bm{E}(\bm{x}) \cdot \bm{n}(\bm{x}) dS = 0
\end{eqnarray*}

ガウスの定理より、

\begin{eqnarray*}
    \oint_{S} \bm{E}(\bm{x}) \cdot \bm{n}(\bm{x}) dS = \int_{V} \nabla \cdot \bm{E}(\bm{x}) d^3 x
\end{eqnarray*}

閉曲面内に点電荷がないので、電場が無限大になるような特異点が無い。よって、電場のdivを普通に計算することができる。

\begin{eqnarray*}
    \frac{\partial}{\partial x} E_{x} 
    &=& \frac{\partial}{\partial x} \frac{e}{4 \pi \epsilon_0} \frac{x - x_{Q}}{R^3} \\
    &=& \frac{e}{4 \pi \epsilon_0} \left\{ \frac{1}{R^3} - \frac{3 ( x - x_{Q} )^2}{R^5} \right\} 
\end{eqnarray*}

同じように y, z の偏微分も計算できるので

\begin{eqnarray*}
    \nabla \cdot \bm{E}(\bm{x}) &=& \frac{e}{4 \pi \epsilon_0} \left\{ \frac{3}{R^3} - \frac{3 ( x - x_{Q} )^2 + 3 ( y - y_{Q} )^2 + 3 ( z - z_{Q} )^2}{R^5} \right\} \\
    &=& \frac{e}{4 \pi \epsilon_0} \left\{ \frac{3}{R^3} - \frac{3 R^2}{R^5} \right\} \\
    &=& \frac{e}{4 \pi \epsilon_0} \left\{ \frac{3}{R^3} - \frac{3}{R^3} \right\} \\
    &=& 0
\end{eqnarray*}

になり、積分しても$0$になる。電場の湧き出しが無いので、閉曲面を出入りする電場が差し引きでゼロになるイメージ。
\end{document}